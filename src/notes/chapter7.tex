%! Author = Len Washington III
%! Date = 10/21/24

% Preamble
\documentclass[
	chapter=7,
	title={Quantum Theory {\&} the Electronic Structure of Atoms},
	showanswers=true,
]{chem122notes}

% Packages

% Document
\begin{document}

\section{Outline}\label{sec:outline-7}
\begin{itemize}
	\item Examine the wavelike properties of light (wavelength, frequency, and speed)
	\item Describe particle behavior of light in terms of quantized energy and photons
	\item Line spectra and the Bohr model
	\item Wave behavior of matter and Heisenberg's Uncertainty Principle
	\item Quantum mechanics and atomic orbitals
	\item Representations of orbitals and electron configurations
\end{itemize}

\section{Electromagnetic Radiation}\label{sec:electromagnetic-radiation}
\begin{itemize}
	\item \definition{Electromagnetic Radiation}{A form of energy that has wave characteristics and that propagates throguh a vacuum at the characteristic speed of light $3.00 \times 10^{8}$ m/s.}
	\item Most subatomic particles behave as PARTICLES
	\item A combination of an electric component and
\end{itemize}

\section{Wavelength and Frequency}\label{sec:wavelength-and-frequency}
\begin{itemize}
	\item \definition{Wavelength}{the distance between two adjacent peaks or between two adjacent troughs.}
	\item \definition{Frequency}{the number of times per second that one complete wavelength passes a point.}
\end{itemize}

\section{Wavelength-Frequency Relationship}\label{sec:wavelength-frequency-relationship}
\begin{itemize}
	\item All electromagnetic radiation moves at the same speed, specifically the speed of light. $c = 2.998 \times 10^{8}$ m/s.
	\item The inverse relationship
\end{itemize}

\section{Common Frequency Unit}\label{sec:common-frequency-unit}
\begin{itemize}
	\item Frequency is typically expressed in cycles per second, a unit also called a \emph{hertz (Hz)}. A \emph{hertz} is equivalent to
\end{itemize}

\section{Hot Objects}\label{sec:hot-objects}
\begin{itemize}
	\item Solids emit radiation when heated (referred to as \textit{Blackbody radiation}).
	\item For example, a stove burner glows bright red, while the filament in a tungsten light bulb glows white.
	\item Hotter objects glow more white.
	\item Wavelength distribution of radiation clearly depends on temperature.
\end{itemize}

\section{Quantization of Energy}\label{sec:quantization-of-energy}
\begin{itemize}
	\item An object can gain or lose enegry by absorbing or emitting radiant energy in \underline{discrete} \emph{QUANTA}.
	\item \textbf{Energy of radiation is proportional to frequency}
	\begin{equation}
		E = h \times \nu
		\label{eq:planck-radiation}
	\end{equation}
	where $h=6.626 \times 10^{-34} J\times s$ is Planck's constant
\end{itemize}

\section{Photoelectric Effect}\label{sec:photoelectric-effect}
\begin{itemize}
	\item Shining light on a clean metal surface causes electrons to be rejected.
	\item For example, cesium metal will emit electrons when irradiated by light with a frequency of $4.60 \times 10^{14}$ Hz or greater.
	Electrons from cesium will not be ejected if lower frequencies are used.
	\item Einstein suggested that an incident stream of tiny energy packets (quanta) were responsible for causing electrons to be ejected from the metals surface.
	\item These discrete energy packets/particles are referred to as ``photons''.
	\item Once electrons were ejected, a current could be measured.
\end{itemize}

\end{document}
