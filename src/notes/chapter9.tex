%! Author = Len Washington III
%! Date = 11/6/24

% Preamble
\documentclass[
	chapter=9,
	title={Basic Concepts of Chemical Bonding},
	showanswers=true,
]{chem122notes}

% Packages

% Document
\begin{document}

\section{Outline}\label{sec:outline-9}

\begin{itemize}
	\item Lewis symbols and valence electrons
	\item Ionic bonding - electrostatic attractions between ions of opposite charge
	\item Covalent bonding - sharing of one or more electron pairs between atoms
	\item Bond polarity and electronegativity
	\item Drawing Lewis structures
	\item Resonances structures, exceptions to the octet rule, and strengths of covalent bonds
\end{itemize}

\section{Chemical Bonds}\label{sec:chemical-bonds}
\definition{Chemical bond}{a strong attractive fource that exists between atoms in a molecule.}
The three types of chemical bonds are as follows:
\begin{description}[font=\emph]
	\item[ionic bond] a bond between oppositely charged ions.
	The ions are formed from atoms by transfer of one or more electrons.
	\item[covalent bond]
\end{description}

\section{Lewis Symbols}\label{sec:lewis-symbols}
\begin{itemize}
	\item The \emph{valence electrons}, those that reside in the outermost shell of an atom, are responsible for chemical bonding.
	\item \emph{Lewis symbol} (electron dot symbol) The chemical symbol for an element, with a dot for each valence electron.
	\item Dots are placed on the four sides of the chemical symbol, where each side can accommodate up to two electrons.
\end{itemize}

\section{Ionic Bonding}\label{sec:ionic-bonding}
\begin{itemize}
	\item The combination of sodium metal and chlorine gas results in a violent
	\chemfig{\charge{90=\.}{Li}}
\end{itemize}

\section{Lattice Energy}\label{sec:lattice-energy}
The energy associated with electrostatic interactions is governed by Coulomb's Law:
\begin{equation}
	E_{el} = \frac{\kappa Q_{1}Q_{2}}{d}
	\label{eq:coulombs-law}
\end{equation}
\begin{itemize}
	\item Lattice energy increases with the charge on the ions.
	\item It also increases with decreasing size of ions.
	\item See the worked example entitled~\nameref{sec:magnitudes-of-lattice-energies}.
\end{itemize}

\section{Magnitudes of Lattice Energies}\label{sec:magnitudes-of-lattice-energies}
Which substance would you expect to have the greatest lattice energy, \ce{MgF2}, \ce{CaF2}, or \ce{ZrO2}?

\begin{answer}
	\reaction{
		MgF2(s) -> Mg^{2+}(g) + 2F-(g)
	}
	Because the product of the charge, $Q_{1}Q_{2}$, appears in the numerator of the equation above, the lattice energy will increase dramatically when the charges of the ions increase.
	Thus,
	\reaction{
		MgF2 \ \ Q1&=+2 \ \ Q2 &= -1\\
		CaF2 \ \ Q1&=+2 \ \ Q2 &= -1\\
		ZrO2 \ \ Q1&=+4 \ \ Q2 &= -2\\
	}
	\reaction{CaF2 $<$ MgF2 $<$ ZrO2}
\end{answer}

\begin{table}[H]
	\centering
	\caption{Lattice Energies for Some Ionic Compounds}
	\label{tab:lattice-energies-for-ionic-compounds}
	\begin{tabular}{l l|l l}
		\textbf{Compound} & \textbf{Lattice Energy (kJ/mol)} & \textbf{Compound} & \textbf{Lattice Energy (kJ/mol)}\\
		\hline
		\ce{LiF} & 1030 & \ce{MgCl2} & 2326 \\
		\ce{LiCl} & 834 & \ce{SrCl2} & 2127 \\
		\ce{LiI} & 730 & & \\
		\ce{NaF} & 910 & \ce{MgO} & 3795 \\
		\ce{} &  & \ce{3414} &  \\
		\ce{} &  & \ce{3217} &  \\
		\ce{} &  & & \\
		\ce{} &  & \ce{7547} &  \\
		\ce{} &  & & \\
		\ce{} &  & & \\
		\ce{} &  & & \\
		\ce{} &  & & \\
	\end{tabular}
\end{table}

\section{Covalent Bonding}\label{sec:covalent-bonding}
\begin{itemize}
	\item In covalent bonds, atoms share electrons.
	\item There are several electrostatic interactions in these bonds:
	\begin{itemize}
		\item Attractions between electrons and positive nuclei.
		\item Repulsions between electrons
		\item Repulsions between nuclei
		\item Attractive forces must outweigh the repulsive ones
	\end{itemize}
\end{itemize}

\section{Lewis Structures}\label{sec:lewis-structures}
\begin{itemize}
	\item Consider two Hydrogen atoms coming together to form a covalently bonded \ce{H2} molecule:
	\reaction{ \chemfig{\charge{0=\.}{H}} + \chemfig{\charge{180=\.}{H}} -> }
	\item The \ce{H2} molecule on the right, with its two electrons, exhibits the noble-gas configuration
	\item Consider two chlorine atoms coming together to form a covalently bonded \ce{Cl2} molecule:
	\reaction{ \chemfig{\charge{0=\.,90=\.\.,180=\.\.,270=\.\.}{Cl} + \charge{0=\.\.,90=\.\.,180=\.,270=\.\.}{Cl}} -> }
	\item Each chlorine atom on the
\end{itemize}

\section{Typical Bonding Motifs}\label{sec:typical-bonding-motifs}
Typical bonding motifs above

\end{document}
