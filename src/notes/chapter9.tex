%! Author = Len Washington III
%! Date = 11/6/24

% Preamble
\documentclass[
	chapter=9,
	title={Basic Concepts of Chemical Bonding},
	showanswers=true,
]{chem122notes}
\usepackage{siunitx}

% Packages

% Document
\begin{document}

\section{Outline}\label{sec:outline-9}

\begin{itemize}
	\item Lewis symbols and valence electrons
	\item Ionic bonding - electrostatic attractions between ions of opposite charge
	\item Covalent bonding - sharing of one or more electron pairs between atoms
	\item Bond polarity and electronegativity
	\item Drawing Lewis structures
	\item Resonances structures, exceptions to the octet rule, and strengths of covalent bonds
\end{itemize}

\section{Chemical Bonds}\label{sec:chemical-bonds}
\definition{Chemical bond}{a strong attractive fource that exists between atoms in a molecule.}
The three types of chemical bonds are as follows:
\begin{description}[font=\emph]
	\item[ionic bond] a bond between oppositely charged ions.
	The ions are formed from atoms by transfer of one or more electrons.
	\item[covalent bond]
\end{description}

\section{Lewis Symbols}\label{sec:lewis-symbols}
\begin{itemize}
	\item The \emph{valence electrons}, those that reside in the outermost shell of an atom, are responsible for chemical bonding.
	\item \emph{Lewis symbol} (electron dot symbol) The chemical symbol for an element, with a dot for each valence electron.
	\item Dots are placed on the four sides of the chemical symbol, where each side can accommodate up to two electrons.
\end{itemize}

\section{Ionic Bonding}\label{sec:ionic-bonding}
\begin{itemize}
	\item The combination of sodium metal and chlorine gas results in a violent
	\chemfig{\charge{90=\.}{Li}}
\end{itemize}

\section{Lattice Energy}\label{sec:lattice-energy}
The energy associated with electrostatic interactions is governed by Coulomb's Law:
\begin{equation}
	E_{el} = \frac{\kappa Q_{1}Q_{2}}{d}
	\label{eq:coulombs-law}
\end{equation}
\begin{itemize}
	\item Lattice energy increases with the charge on the ions.
	\item It also increases with decreasing size of ions.
	\item See the worked example entitled~\nameref{sec:magnitudes-of-lattice-energies}.
\end{itemize}

\section{Magnitudes of Lattice Energies}\label{sec:magnitudes-of-lattice-energies}
Which substance would you expect to have the greatest lattice energy, \ce{MgF2}, \ce{CaF2}, or \ce{ZrO2}?

\begin{answer}
	\reaction{
		MgF2(s) -> Mg^{2+}(g) + 2F-(g)
	}
	Because the product of the charge, $Q_{1}Q_{2}$, appears in the numerator of the equation above, the lattice energy will increase dramatically when the charges of the ions increase.
	Thus,
	\reaction{
		MgF2 \ \ Q1&=+2 \ \ Q2 &= -1\\
		CaF2 \ \ Q1&=+2 \ \ Q2 &= -1\\
		ZrO2 \ \ Q1&=+4 \ \ Q2 &= -2\\
	}
	\reaction{CaF2 $<$ MgF2 $<$ ZrO2}
\end{answer}

\begin{table}[H]
	\centering
	\caption{Lattice Energies for Some Ionic Compounds}
	\label{tab:lattice-energies-for-ionic-compounds}
	\begin{tabular}{l l|l l}
		\textbf{Compound} & \textbf{Lattice Energy (kJ/mol)} & \textbf{Compound} & \textbf{Lattice Energy (kJ/mol)}\\
		\hline
		\ce{LiF} & 1030 & \ce{MgCl2} & 2326 \\
		\ce{LiCl} & 834 & \ce{SrCl2} & 2127 \\
		\ce{LiI} & 730 & & \\
		\ce{NaF} & 910 & \ce{MgO} & 3795 \\
		\ce{} &  & \ce{3414} &  \\
		\ce{} &  & \ce{3217} &  \\
		\ce{} &  & & \\
		\ce{} &  & \ce{7547} &  \\
		\ce{} &  & & \\
		\ce{} &  & & \\
		\ce{} &  & & \\
		\ce{} &  & & \\
	\end{tabular}
\end{table}

\section{Covalent Bonding}\label{sec:covalent-bonding}
\begin{itemize}
	\item In covalent bonds, atoms share electrons.
	\item There are several electrostatic interactions in these bonds:
	\begin{itemize}
		\item Attractions between electrons and positive nuclei.
		\item Repulsions between electrons
		\item Repulsions between nuclei
		\item Attractive forces must outweigh the repulsive ones
	\end{itemize}
\end{itemize}

\section{Lewis Structures}\label{sec:lewis-structures}
\begin{itemize}
	\item Consider two Hydrogen atoms coming together to form a covalently bonded \ce{H2} molecule:
	\reaction{ \chemfig{\charge{0=\.}{H}} + \chemfig{\charge{180=\.}{H}} -> }
	\item The \ce{H2} molecule on the right, with its two electrons, exhibits the noble-gas configuration
	\item Consider two chlorine atoms coming together to form a covalently bonded \ce{Cl2} molecule:
	\reaction{ \chemfig{\charge{0=\.,90=\:,180=\:,270=\:}{Cl} + \charge{0=\:,90=\:,180=\.,270=\:}{Cl}} -> \chemfig{\charge{0=\:,90=\:,180=\:,270=\:}{Cl}\charge{0=\:,90=\:,270=\:}{Cl}}}
	\item Each chlorine atom on the right now has a \textit{complete octet} of electrons by sharing the bonding electron pair.
	It achieves the noble gas configuration of argon (Ar).
	Again, the shared pair of electrons can be represented by a single bond, as shown below.
\end{itemize}

\section{Typical Bonding Motifs}\label{sec:typical-bonding-motifs}
Typical bonding motifs above

\section{Bond Polarity and Electronegativity}\label{sec:bond-polarity-and-electronegativity}
\begin{itemize}
	\item Molecules such as \ce{H2}, \ce{N2}, \ce{Cl2}, etc are said to be \textbf{nonpolar}.
	\item \emph{A nonpolar covalent bond} is one in which the electrons are shared equally between two atoms.
	\item On the other hand, \emph{a polar covalent bond} is one in which one of the atoms exerts a greater attraction for the bonding electrons than the other.
	\item In other words, there exists a bond between atoms of different \emph{electronegativities}.
\end{itemize}

\section{Electronegativity}\label{sec:electronegativity}
\begin{itemize}
	\item \definition{Electronegativity}{the ability of atoms}
	\item On the periodic table
\end{itemize}

\begin{table}[H]
	\centering
	\caption{Electronegativity and Bond Polarity}
	\label{tab:electronegativity}
	\begin{tabular}{|*{4}{c|}}
		\hline
		\textbf{Compound} & \textbf{\ce{F2}} & \textbf{\ce{HF}} & \textbf{\ce{LiF}}\\
		\hline
		Electronegativity & 4.0 - 4.0 = 0 & 4.0 - 2.1 = 1.9 & 4.0 - 1.0 = 3.0\\
		\hline
		Type of bond & Nonpolar covalent & Polar covalent & Ionic\\
		\hline
	\end{tabular}
\end{table}

\begin{table}[H]
	\centering
	\caption{Polar Covalent Bonds}
	\label{tab:polar-covalent-bonds}
	\begin{tabular}{p{0.2\textwidth} p{0.22\textwidth} p{0.22\textwidth} p{0.25\textwidth}}
		\textbf{Compound} & \textbf{Bond Length (\si{\angstrom})} & \textbf{Electronegativity} & \textbf{Dipole Moment (D)}\\
		\hline
		\ce{HF}  & 0.92 & 1.9 & 1.82\\
		\ce{HCl} & 1.27 & 0.9 & 1.08\\
		\ce{HBr} & 1.41 & 0.7 & 0.82\\
		\ce{HI}  & 1.61 & 0.4 & 0.44\\
		\hline
	\end{tabular}
\end{table}

\section{Writing Lewis Structures}\label{sec:writing-lewis-structures}
\reaction{\chemfig{H-H}}
\reaction{\chemfig{\charge{90=\:,180=\:,270=\:}{Cl}-\charge{0=\:,90=\:,270=\:}{Cl}}}

\begin{enumerate}
	\item
	\item The central atom is the \emph{least} electronegative element that isn't Hydrogen.
	Connect the other atoms to it by single bonds.
	\item Fill the octets of the outer atoms.
	\begin{itemize}
		\item \textcolor{red}{How many electrons have you accounted for in the above structure?} \begin{answer}24\end{answer}
		\item \textcolor{red}{How many do you have left?} \begin{answer}2\end{answer}
		\item Fill in the octet of the central atom.
		\item If you run out of electrons before the central atom has an octet: form multiple bonds until it does
		\reaction{\chemfig{H-C-\charge{0=\:,90=\:,270=\:}{N}} -> \chemfig{H-C~\charge{0=\:}{N}}}
	\end{itemize}
\end{enumerate}

\section{Lewis Structures for Polyatomic Ions}\label{sec:lewis-structures-for-polyatomic-ions}
Draw the Lewis structures for:
\begin{enumerate}[label=(\alph*)]
	\item \ce{ClO2-}
	\begin{answer}
	\end{answer}
	\item \ce{SO4^2-}
	\begin{answer}
		6 + 4(6) + 2 = 32 valence electrons
		\reaction{\chemfig{\charge{90=\:,180=\:,270=\:}{O}-S(-[:90]\charge{0=\:,90=\:,180=\:}{O})(-[:270]\charge{0=\:,180=\:,270=\:}{O})-\charge{0=\:,90=\:,270=\:}{O}}}
	\end{answer}
	\item \ce{CO3^2-}
	\begin{answer}
		4 + 3(6) + 2 = 24 valence electrons
		\begin{equation*}
			\chemfig{([:30]\charge{135=\:,225=\:,315=\:}{O}-) C (=[:90]\charge{45=\:,135=\:}{O})(-[:-30]\charge{45=\:,225=\:,315=\:}{O})} \hfill\ \ \ \ \ \ \ \ \ \ \ \ \ \
			\chemfig{([:30]\charge{135=\:,315=\:}{O}=) C (-[:90]\charge{0=\:,90=\:,180=\:}{O})(-[:-30]\charge{45=\:,225=\:,315=\:}{O})} \hfill\ \ \ \ \ \ \ \ \ \ \ \ \ \
			\chemfig{([:30]\charge{135=\:,225=\:,315=\:}{O}-) C (-[:90]\charge{0=\:,90=\:,180=\:}{O})(=[:-30]\charge{45=\:,225=\:}{O})}
		\end{equation*}
	\end{answer}
\end{enumerate}

\section{Resonance}\label{sec:resonance}

\section{Exceptions to the Octet Rule}\label{sec:exceptions-to-the-octet-rule}
\begin{itemize}
	\item The three types of systems that don't follow the octet rule are as follows:
	\begin{itemize}
		\item Ions or molecules with an odd number of electrons
		\item Ions or molecules with less than an octet
		\item Ions or molecules with more than eight valence electrons (an expanded octet)
	\end{itemize}
\end{itemize}

\end{document}
