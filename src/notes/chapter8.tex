%! Author = Len Washington III
%! Date = 11/6/24

% Preamble
\documentclass[
	chapter=8,
	title={Periodic Properties of the Elements},
	showanswers=true,
]{chem122notes}

% Packages

% Document
\begin{document}

\section{Outline}\label{sec:outline-8}
\begin{itemize}
	\item
\end{itemize}

\section{Trends in First Ionization Energies}\label{sec:trends-in-first-ionization-energies}
\begin{itemize}
	\item As one goes down a group, less energy is required to remove the first electron.
	\begin{itemize}
		\item For atoms in the same group, $Z_{eff}$ is essentially the same, but the valence electrons are farther from than $\dots$
	\end{itemize}
	\item Generally, as one goes across a row/period, it becomes more difficult to remove an electron.
	\begin{itemize}
		\item As you go from left to right $\rightarrow Z_{eff}$ increases!
	\end{itemize}
\end{itemize}

Account for the decrease in ionization energy in going from nitrogen (N) to oxygen (O) despite the increase in effective nuclear charge ($Z_{eff}$).

\begin{answer}
\end{answer}

\section{Electron Affinity}\label{sec:electron-affinity}
Electron affinity is the energy change accompanying the addition of an electron to a gaseous atom:
\begin{equation*}
\begin{aligned}
	\ce{CL(g) + e- ->  Cl-(g)}\ \ \ \ \ E_{a} = -349 \frac{\mbox{kJ}}{\mbox{mol}}
\end{aligned}
\end{equation*}
Energy is typically released when an electron is added to a gaseous atom.
The process is said to be \emph{exothermic}, so the energy has a negative sign associated with it.\\

The electron affinity of lithium is a negative value, whereas the electron affinity of Beryllium is a positive value.
Use electron configuration to account for this observation.

\end{document}
