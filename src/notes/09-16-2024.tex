%! Author = Len Washington III
%! Date = 9/16/24

% Preamble
\documentclass{chem122notes}

% Packages

% Document
\begin{document}

\section{Outline}\label{sec:outline}
\begin{itemize}
	\item 
\end{itemize}

\section{Terminology}\label{sec:terminology}
In solution, we need to define the following terms:
\begin{itemize}
	\item \definition{solvent}{The medium (e.g., water, ethanol, benzene, etc.) in which a solute is dissolved to form a solution.}
	\item \definition{solute}{The substance (e.g. NaCl, glucose, etc.) dissolved in a solvent to form a solution.}
\end{itemize}

\section{Concentration of Solute}\label{sec:concentration-of-solute}
The amount of solute in a solution is given by its \emph{concentration}.
\begin{equation}
	\mbox{Molarity(M)} = \frac{\mbox{moles of solute}}{\mbox{liters of solution}}
	\label{eq:molarity}
\end{equation}

\begin{equation*}
\begin{aligned}
	[\mbox{NaCl}] &= 0.1 M\\
		      &= \frac{0.1 \mbox{ moles of NaCl}}{1 \mbox{ L of solution}}
\end{aligned}
\end{equation*}

\section{Preparing Solutions}\label{sec:preparing-solutions}
\begin{itemize}
	\item Weigh out a solid solute and dissolve in a given quanitity of solvent.
	\item \emph{Dilute} a concentrated solution to give one that is less concentrated.
\end{itemize}

\section{Using Molarity, }
What mass of oxalic acid, $\chemical{H}[2]\chemical{C}[2]\chemical{O}[4]$, is required to make 250.00 mL of a 0.0500 M solution?

\begin{equation*}
\begin{aligned}
	\mbox{molar mass} &= (2*1.008) + (2*12.011) + (4 * 15.999)\\
			  &= 2.016 + 24.022 + 63.996\\
			  &= 90.034 \mbox{ g mol}^{-1}\\
\end{aligned}
\end{equation*}
\begin{equation*}
\begin{aligned}
	250.00 \mbox{ mL} \times \frac{1 \mbox{ L}}{1,000 \mbox{ mL}} \times \frac{0.05 \mbox{ mol}}{1 \mbox{L}} \times \frac{90.034 \mbox{ g }\chemical{H}[2]\chemical{C}[2]\chemical{O}[4]}{1 \mbox{ mol } \chemical{H}[2]\chemical{C}[2]\chemical{O}[4]} = 1.125425 \mbox{ g }\chemical{H}[2]\chemical{C}[2]\chemical{O}[4]\\
\end{aligned}
\end{equation*}

\section{Preparing a Solution by Dilution}\label{sec:preparing-a-solution-by-dilution}
You have 50.0 mL of 3.0 M NaOH and you want 0.50 M NaOH\@.
What does one do?

\begin{equation*}
\begin{aligned}
	M_{1}V_{1} &= M_{2}V_{2}\\
	(3.0)(50.0) &= (0.5)V_{2}\\
	V_{2} &= \frac{(3.0)(50.0)}{0.5}\\
	V_{2} &= (6.0)(50.0)\\
	V_{2} &= 300.0\mbox{ mL}\\
		  &= 3.0\times10^{2}\mbox{ mL}
\end{aligned}
\end{equation*}

In an acid-base titration, it takes 38.55 mL of 0.650 M perchloric acid (HCl$\chemical{O}[4]$) to completely neutralize 25.00 mol calcium hydroxide ($\chemical{Ca}(\chemical{O}\chemical{H}[2])$) solution.
\[ \chemical{Ca}(\chemical{O}\chemical{H}[2]) (\mbox{aq.}) + 2\chemical{H}\chemical{Cl}\chemical{O}[4] (\mbox{aq.}) \rightarrow \chemical{Ca}( \chemical{Cl}{O_{4}} )_{2} (\mbox{aq.}) + 2\chemical{H}[2]\chemical{O} (\mbox{l}) \]
\begin{enumerate}[label=\Alph*)]
	\item How many moles of $\chemical{H}\chemical{Cl}\chemical{O}[4]$ are needed for the complete neutralization?
	\begin{equation*}
	\begin{aligned}
		38.55\mbox{mL perchloric acid} \times \frac{1 \mbox{ L perchloric acid}}{1,000 \mbox{ mL perchloric acid}} \times \frac{0.650 \mbox{ mol}}{1 \mbox{ L}} = 0.0251 \mbox{ mol perchloric acid}
	\end{aligned}
	\end{equation*}
	\item How many moles of $\chemical{Ca}(\chemical{O}\chemical{H}[2])_{2}$ got consumed during the neutralization?
	\begin{equation*}
	\begin{aligned}
		0.0251 \mbox{ mol perchloric acid} \times \frac{1 \mbox{ mol } \element{Ca}(\element{O}\element{H}[2])_{2}}{2 \mbox{ mol perchloric acid}} &= 0.01255 \ \element{Ca}(\element{O}\element{H}[2])_{2}
	\end{aligned}
	\end{equation*}
	\item What is the concentration of $\element{Ca}(\element{O}\element{H}[2])_{2}$ in the original solution before titration?
	\begin{equation*}
	\begin{aligned}
		\frac{25.00 \mbox{ mol } \element{Ca}(\element{O}\element{H}[2])_{2}}{38.55 \mbox{mL}} \times \frac{1,000 \mbox{ mL}}{1 \mbox{ mL}} =
	\end{aligned}
	\end{equation*}
\end{enumerate}

\section{Dissociation}\label{sec:dissociation}
\begin{itemize}
	\item When an ionic compound dissolves in water, the solvent pulls the individual ions from the crystal and solvates them.
	\item This process is called \emph{dissociation}.
\end{itemize}
\[ \element{Na}\element{Cl} (\mbox{s}) \rightarrow \element{Na}^{+} (\mbox{aq}) + \element{Cl}^{-}(\mbox{aq}) \]
\begin{itemize}
	\item An \emph{electrolyte} is a substance that dissociates into ions when dissolved in water.
	\item Ionic compounds dissociate in water ().
	\item Only a few molecular compounds are capable of dissociating in water.
	\item For example, \[ \chemical{H}\chemical{Cl} \rightarrow \chemical{H}^{+} + \chemical{Cl}^{-} \]
\end{itemize}

\section{Electrolytes}\label{sec:electrolytes}
\begin{itemize}
	\item An \emph{electrolyte} is a substance that dissociates into ions when dissolved in water.
	\item A \emph{nonelectrolyte} may dissolve in water, but it does not dissociate into ions when it does so.
	\item There are many examples of molecular compounds (e.g., ) that serve as nonelectrolytes in water.
\end{itemize}

\begin{itemize}
	\item Ion concentration can be measured using conductivity.
	\begin{description}
		\item[No ions] A nonelectrolyte solution does not contain ions, and the bulb does not light.
		\item[Few ions] If the solution contains a small number of ions
	\end{description}
\end{itemize}

\end{document}
