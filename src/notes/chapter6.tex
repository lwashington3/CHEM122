%! Author = Len Washington III
%! Date = 10/14/24

% Preamble
\documentclass[
	chapter=6,
	title={Thermochemistry},
	showanswers=true,
]{chem122notes}

% Packages

% Document
\begin{document}

Chapters 1-5 study matter, now we study energy.

Warming your hands with chemical hand warmers involves many of the principles of \emph{thermochemistry}, the study of the relationships between chemistry and energy.
When you open the package that contains the hand warmer, the contents are exposed to air, and a reaction that gives off heat to its surroundings occurs.
Most handwarmers involve the oxidation of iron:
\[ \ce{4Fe(s) + 3O2(g) -> 2Fe2O3(s)} \]

In this chapter, we look at how chemical reactions can exchange energy with their surroundings and how we can quantify the magnitude of those exchanges.

\subsection{Applications}\label{subsec:applications}
\begin{itemize}
	\item Heating of homes
	\item Production of energy
\end{itemize}

\section{Key Definitions}\label{sec:key-definitions}
\begin{description}
	\item[Energy] Capacity to do work
	\item[Work] Result of a force active through a distance
	\item[Examples of work]~
	\begin{itemize}
		\item Pushing a box across the floor
		\item a billiard ball rolling across a billiard table and colliding with a second, stationary ball
	\end{itemize}
	\item[Potential energy] Associated with position or composition. Example: Raising a billiard ball off the table increases its potential energy.
	\item[Chemical energy] Associated with relative positions of electrons and nuclei in atoms and molecules
	% TODO: Add Tikz representation of tree
	\item[Law of conservation of energy] energy can be neither created nor destroyed; it can assume different forms
	\item[System] chemicals in a beaker (or handwarmers) for example
	\item[Surrounding] water that the chemicals are dissolved in, the beaker, the lab bench, air in the room, etc.
	\begin{itemize}
		\item Surroundings gain the exact amount of energy lost by the system and vice versa.
	\end{itemize}
\end{description}

\section{Units of Energy}\label{sec:units-of-energy}
\begin{itemize}
	\item $KE = \frac{1}{2}mv^{2}$, $[KE] = [m][v] = \mbox{kg}\times \frac{m}{s}$
\end{itemize}

\section{sec:1st-law-of-thermodynamics}\label{sec:sec:1st-law-of-thermodynamics}
\begin{itemize}
	\item The total energy of the universe is constant $\rightarrow$ Energy is neither created, nor destroyed, universe does not exchange energy with anything else.
	\item According to the 1st law, a device that continually produces energy with no energy input cannot exist.
\end{itemize}

\subsection{Internal Energy (IE)}\label{subsec:internal-energy-(ie)}
\begin{itemize}
	\item The internal energy of a system is the sum of the kinetic and potential energies of all the particles that compose the systems.
	\item It is a ``state function''.
	\item State of a chemical system is specified by parameters such as temperature, pressure, concentration, and phase (solid, liquid, or gas)
	\item Elevation of 10,000 ft, for example, is a state function no matter how you climbed it;
	the distance, however, is not a state function as you can take any route.
\end{itemize}


\end{document}
