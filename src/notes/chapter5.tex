%! Author = Len Washington III
%! Date = 10/2/24

% Preamble
\documentclass[
	chapter=5,
	title={Gases},
	showanswers=true,
]{chem122notes}

% Packages

% Document
\begin{document}

\section{Characteristics of Gases}\label{sec:characteristics-of-gases}
\begin{itemize}
	\item Gas molecules are very far apart, so total volume of gases is mostly empty space.
	\item Gas molecules move at high velocities and high kinetic energies.
	\item Gas volume is dependent on pressure, temperature and moles of the gas.
\end{itemize}

\section{Real Gases and Ideal Gases}\label{sec:real-gases-and-ideal-gases}
\begin{itemize}
	\item Real gases obey the laws of ``ideal'' gases at non-extreme conditions.
	\item Therefore, gas behavior (pressure, temperature, volume, and number of particles) is determined using gas laws.
\end{itemize}

\section{Pressure}\label{sec:pressure}
\begin{itemize}
	\item Pressure is the force per unit area.
	\begin{equation}
		P = \frac{Force}{Area} \ \ or\ \ P = \frac{F}{A}\\
		\label{eq:pressure}
	\end{equation}
	\item So, the pressure will increase with a greater force exerted on an area or with a certain force exerted on a smaller area.
\end{itemize}

\section{Measuring Pressure}\label{sec:measuring-pressure}
\begin{minipage}[m]{0.7\textwidth}
	\begin{itemize}
		\item A barometer is one way to measure pressure.
		\item The height of the mercury column in the tube correlates to the atmospheric pressure.
	\end{itemize}
\end{minipage}\hfill%
\begin{minipage}[m]{0.3\textwidth}
	% TODO: Insert the image
\end{minipage}

\section{Pressure Units}\label{sec:pressure-units}
\begin{itemize}
	\item There are several different units used to describe pressure. We will focus on four of them:
	\begin{description}[font=\bfseries]
		\item[atm] atmospheres
		\item[mmHg] millimeters of Mercury
		\item[torr]
		\item[psi] pounds per square inch
	\end{description}
	\item 1 atm = 760 mmHg, 1 atm = 760 torr, and 1 atm = 14.7 psi
	\item So 1 mmHg = 1 torr.
	\begin{itemize}
		\item Note: atm, mmHg, and torr are all exact values; psi has been rounded to three sig figs.
	\end{itemize}
\end{itemize}

\section{Examples}\label{sec:examples}
\begin{enumerate}[label=\arabic*.]
	\item A tire pressure gauge reads 33 psi. What is this pressure reading in torr?%
	\begin{answer}
		\begin{equation*}
		\begin{aligned}
			33 \mbox{ psi} \times \frac{1 \mbox{ atm}}{14.7 \mbox{ psi}} \times \frac{\mbox{760 torr}}{ \mbox{1 atm}} = 1,\underline{7}06.12245 \mbox{ torr}
		\end{aligned}
		\end{equation*}
	\end{answer}
	\item In a near-vacuum, the pressure is 0.100 mmHg. What is this pressure in atmospheres?%
	\begin{answer}
		\begin{equation*}
		\begin{aligned}
			0.100 \mbox{ mmHg} \times \frac{1 \mbox{ atm}}{760 \mbox{ mmHg}} = 1.31579 \times 10^{-4} \mbox{ atm}
		\end{aligned}
		\end{equation*}
	\end{answer}
\end{enumerate}

\section{Temperature}\label{sec:temperature}
\begin{itemize}
	\item One factor involved in gas behavior is temperature (a measure of hot and cold).
	\item In the lab, temperature is measured in \textdegree{}C + 273.15
	\item \textbf{Lecture Problem:} A gas is collected in the laboratory at 34\textdegree{}C\@. What is the temperature on the Kelvin scale?
	\begin{answer}
		34\textdegree{}C + 273.15 = 307.15\textdegree K
	\end{answer}
\end{itemize}

\section{The Gas Laws}\label{sec:the-gas-laws}
\begin{itemize}
	\item The Gas Laws focus on the relationship between pressure, temperature, volume, and the amount of a gas.
	\item We will discuss four gas laws in these sections: \hyperref[sec:boyles-law]{Boyle's Law}, Charles' Law, Amonton's Law, and the Combined Gas Law.
	\item Note: Any references to standard temperature and pressure (STP\label{dfn:stp}) means 273 \textdegree{}K and 1 atm.
\end{itemize}

\section{Boyle's Law}\label{sec:boyles-law}
\begin{itemize}
	\item If the temperature and amount of a gas are held constant, then the volume of a gas will be inversely proportional to its pressure.
\end{itemize}
\begin{equation}
	\mbox{\textbf{P}}\uparrow\mbox{\textbf{V}}\downarrow \mbox{  and  } \mbox{\textbf{P}}\downarrow\mbox{\textbf{V}}\uparrow
	\label{eq:boyles-law}
\end{equation}

\subsection{Boyle's Law Mathematically}\label{subsec:boyle's-law-mathematically}
Since volume and pressure are inversely proportional to each other, then the mathematical relationship would be:
\[ V = \frac{k}{P} \mbox{  or  } PV = k \mbox{   where } k \mbox{ is a constant.} \]

\end{document}
